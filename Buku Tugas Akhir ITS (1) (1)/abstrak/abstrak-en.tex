\begin{center}
  \large\textbf{ABSTRACT}
\end{center}

\addcontentsline{toc}{chapter}{ABSTRACT}

\vspace{2ex}

\begingroup
% Menghilangkan padding
\setlength{\tabcolsep}{0pt}

\noindent
\begin{tabularx}{\textwidth}{l >{\centering}m{3em} X}
  \emph{Name}     & : & \name{}         \\

  \emph{Title}    & : & \engtatitle{}   \\

  \emph{Advisors} & : & 1. \advisor{}   \\
                  &   & 2. \coadvisor{} \\
\end{tabularx}
\endgroup

% Ubah paragraf berikut dengan abstrak dari tugas akhir dalam Bahasa Inggris
\emph{This study aims to develop a speech notification generation system for service robots using
a multimodal sensor-based Language Model (LLM). By leveraging sensor technology capable of capturing diverse inputs, such as sound, motion, and visual cues, robot services can effectively communicate with users through more contextual and responsive speech notifications.
The LLM serves as the core language processor to enhance language understanding and response generation in the robot. The system is expected to improve human-machine interaction
by enabling robots to provide notifications in a more intuitive and adaptive manner, creating a more satisfying user experience. The research methodology involves the development and
integration of multimodal sensor technology, the implementation of LLM, and the evaluation of system performance through simulation tests and direct testing. The outcomes of this research are anticipated to contribute to the advancement of sophisticated and efficient human-robot interaction technology.}

% Ubah kata-kata berikut dengan kata kunci dari tugas akhir dalam Bahasa Inggris
\emph{Keywords}: \emph{Service Robot}, \emph{Speech Notification}, \emph{LLM}, \emph{Multimodal Sensor}, \emph{HUman-Robot Interaction}, \emph{Natural Language Processing}.
