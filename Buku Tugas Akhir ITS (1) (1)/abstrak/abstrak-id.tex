\begin{center}
  \large\textbf{ABSTRAK}
\end{center}

\addcontentsline{toc}{chapter}{ABSTRAK}

\vspace{2ex}

\begingroup
% Menghilangkan padding
\setlength{\tabcolsep}{0pt}

\noindent
\begin{tabularx}{\textwidth}{l >{\centering}m{2em} X}
  Nama Mahasiswa    & : & \name{}         \\

  Judul Tugas Akhir & : & \tatitle{}      \\

  Pembimbing        & : & 1. \advisor{}   \\
                    &   & 2. \coadvisor{} \\
\end{tabularx}
\endgroup

% Ubah paragraf berikut dengan abstrak dari tugas akhir
Penelitian ini bertujuan untuk mengembangkan sistem pembangkitan notifikasi ucapan
pada robot service menggunakan Language Model (LLM) berbasis sensor multimodal. Dengan
memanfaatkan teknologi sensor yang mampu menangkap beragam input, seperti suara, gerakan,
dan visual, robot service dapat secara efektif berkomunikasi dengan pengguna melalui noti-
fikasi ucapan yang lebih kontekstual dan responsif. LLM digunakan sebagai inti pengolah ba-
hasa untuk meningkatkan kemampuan pemahaman dan generasi respons pada robot. Sistem ini
diharapkan dapat meningkatkan interaksi manusia-mesin dengan memungkinkan robot mem-
berikan notifikasi dengan lebih intuitif dan adaptif, menciptakan pengalaman pengguna yang
lebih memuaskan. Metodologi penelitian melibatkan pengembangan dan integrasi teknologi
sensor multimodal, implementasi LLM, serta evaluasi kinerja sistem melalui uji coba simulasi
dan pengujian langsung. Hasil dari penelitian ini diharapkan dapat memberikan kontribusi pada
pengembangan teknologi interaksi manusia-mesin yang lebih canggih dan efisien.

% Ubah kata-kata berikut dengan kata kunci dari tugas akhir
Kata Kunci: Robot \emph{Service}, Notifikasi Ucapan, \emph{LLM}, Sensor Multimodal, Interaksi Manusia-Robot, Pemrosesan Bahasa Alami.
