% Atur variabel berikut sesuai namanya

% nama
\newcommand{\name}{Aryaduta Putra Perkasa}
\newcommand{\authorname}{Musk, Elon Reeve}
\newcommand{\nickname}{Aryaduta}
\newcommand{\advisor}{ Prof. Dr. Ir. Mauridhi Hery Purnomo, M.Eng.}
\newcommand{\coadvisor}{Muhtadin, S.T., M.T.}
\newcommand{\examinerone}{Eko Pramunanto, S.T., M.T.}
\newcommand{\examinertwo}{Ir. Hany Boedinugroho, M.T.}
\newcommand{\examinerthree}{Ahmad Zaini, S.T., M.Sc.}
\newcommand{\headofdepartment}{Dr. Supeno Mardi Susiki Nugroho, S.T., M.T.}

% identitas
\newcommand{\nrp}{5024201077}
\newcommand{\advisornip}{19580916 198601 1 001}
\newcommand{\coadvisornip}{19810609 200912 1 003}
\newcommand{\examineronenip}{19661203199412 1 001}
\newcommand{\examinertwonip}{19610706 198701 1 001}
\newcommand{\examinerthreenip}{197504192 00212 1 003}
\newcommand{\headofdepartmentnip}{197003131 99512 1 0011}

% judul
\newcommand{\tatitle}{PEMBANGKITAN NOTIFIKASI UCAPAN PADA ROBOT SERVICE MENGGUNAKAN LLM BERBASIS SENSOR MULTIMODAL }
\newcommand{\engtatitle}{\emph{SPEECH NOTIFICATION GENERATION FOR SERVICE ROBOTS USING MULTIMODAL SENSOR-BASED LLM}}

% tempat
\newcommand{\place}{Surabaya}

% jurusan
\newcommand{\studyprogram}{Teknik Komputer}
\newcommand{\engstudyprogram}{Computer Engineering}

% fakultas
\newcommand{\faculty}{Teknologi Elektro dan Informatika Cerdas}
\newcommand{\engfaculty}{Intelligent Electrical and Informatics Technology}

% singkatan fakultas
\newcommand{\facultyshort}{FTEIC}
\newcommand{\engfacultyshort}{F-ELECTICS}

% departemen
\newcommand{\department}{Teknik Komputer}
\newcommand{\engdepartment}{Computer Engineering}

% kode mata kuliah
\newcommand{\coursecode}{EC234801}
